\chapter{引言}

目前的射击打靶训练,基本以实弹训练为主,国防开支大,危险系数高。传统的报靶方法是人工报靶,由报靶员根据经验确定靶数,带有很大的个人主观因素,可靠性、公正性差,效率低。因此有必要研制一种切合部队实际的,在非实弹射击条件下进行射击精度训练的打靶训练器,这样既能保证部队训练质量又能减少弹药消耗、节约国防费用,具有重大的国防意义。

以光代弹,可以模拟多种武器的射击情况,并可检验射击效果。这种新型的部队训练模拟器材是部队训练器材的一次革命,是和平时期部队训练的有效手段之一。一些发达国家,如美国、英国、德国等都在积极进行激光射击模拟训练器材的研制,并已开发出多种系列产品,其中最突出的是美国的“米勒斯”系列,它可模拟 36 种武器,性能好、准确而且逼真,大大推动了部队的训练工作。

八十年代以来,我国也有单位在进行激光模拟训练器的研究和探索,将激光射击模拟器用于部队训练,取得了很好的训练效果,提高了部队的战斗力。但在可靠性和数据处理等方面尚有许多技术问题有待改进,主要是以下几点:激光光斑太大,与实际步枪子弹口径$\qty{7.62}\mm$相差太多;探测器数量少会导致设计精度不高;探测器数量多会使得价格昂贵,无法推广;只能粗略指示命中与否,不能准确显示命中靶环环数和方位。因此,我们拟从这些方向作进一步的研究探索。

本设计采用半导体激光器和半导体面阵列探测器来模拟子弹射击和射击靶标,具有模拟逼真,精度高等特点。主要从信号处理部分来设计实现激光打靶系统,每次射击能精确的显示 5 -- 10 环的结果及脱靶情况,每个环数又可分为八个偏移方向。该系统简单实用,既能保证训练的质量又能减少弹药的消耗,是理想的公安、军队等部门训练使用的模拟打靶系统。