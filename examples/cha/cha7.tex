\chapter{结论}

本设计方案达到了任务书的要求,实现了激光信号的检测、编码和串行传输,实现了较为完整的激光打靶系统的信号处理:
\begin{enumerate}
  \item 前端放大器的放大倍数适中,放大后,有效电压脉冲的幅度达到施密特触发器的上门限电压,背景干扰信号没有引起电路的误响应;
  \item 经过调试,实现 40-6 优先编码器的扩展,编码值输出符合真值表,编码有效脉冲下降沿的波形正常;
  \item 由开关按钮(模拟激光枪的扳机)控制的编码采集和串行传送也调试实现(通过与计算机的串口相连,用``串口调试程序''调试);
  \item 信号处理电路通过串口连接到计算机,应用张雪荣同学设计的``激光打靶成绩统计''软件进行总体调试,实现对打靶成绩的显示统计和储存。
\end{enumerate}

由于时间、水平和经验有限,在信号的放大、编码及抗干扰等方面仍有不足之处,有改进的余地,比如电路规模的精简,其他的光干扰处理。另外在系统的调试方面,由于时间和设备的原因,只进行了短距离的调试,有待进一步的调试。这次毕业设计对于我来说,既是一次机遇,又是一次挑战。通过这次的毕业设计,我学到了很多东西,通过自己的实践,增强了动手能力。通过实际工程的设计也使我了解到书本知识和实际应用的差别。在实际应用中遇到很多的问题,这都需要我对问题进行具体的分析,并一步一步地去解决它。