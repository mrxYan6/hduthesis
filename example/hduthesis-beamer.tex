\documentclass [ svgnames,aspectratio = 2013, handout ] {beamer}

\usetheme{hdu}

\infoset
  {
    title      = Research and Application of Micromagnetic Simulation Based on Landau-Lifshitz-Gilbert Equation,
    subtitle   = hdu Undergraduate Thesis Proposal,
    author     = SAN Chi Nan (C668668E0),
    date       = {\today{} / Building 6, Room 321},
    supervisor = Prof. YIP Tsz Ching,
    % reference  = reference.bib,
  }

\begin{document}

\maketitle

\section{Research Methods}

\begin{frame}{Landau-Lifshitz-Gilbert Equation}
  \pause
  Landau-Lifshitz-Gilbert (LLG) equation describes the microkinetics of magnetization in ferromagnetic materials. It combines the Landau-Lifshitz (LL) equation and the Gilbert damping term $\alpha$, which is used to simulate and understand the micro-magnetic dynamics phenomena such as the motion of magnetic domain walls and magnetization reversal.
  \pause
  \begin{equation}
    \odv{\mathbf m}{t} = -\gamma \mathbf m \times \mathbf H_\text{eff} -
    \boxed{\alpha \mathbf m \times \odv{\mathbf m}{t}}
  \end{equation}
  \pause
  To process the term $\alpha \mathbf m \times \odv{\mathbf m}/{t}$,
  we left multiply the LLG equation by $\mathbf m$ and use the identity
  $\mathbf m \cdot \odv{\mathbf m}/{t} = 0$ to generate LL equation.
  \pause
  \begin{equation}
    \odv{\mathbf m}{t} = -\frac{\gamma}{1 + \alpha^2} \mathbf m \times \mathbf H - \frac{\gamma\alpha}{1 + \alpha^2} \mathbf m \times \mathbf m \times \mathbf H
  \end{equation}
  \pause
  \alert{The LLG equation is more convenient for numerical calculation, while the LL equation can introduce the dissipation term more physically.}
\end{frame}

\end{document}