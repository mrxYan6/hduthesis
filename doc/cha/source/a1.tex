\subsection{\file{hduthesis.cls} 的实现}

文档类日期/版本号/开发者id
\begin{minted} [ linenos, bgcolor = bg, breaklines ] {tex}
  \def\hduthesis@date{2024/12/23}
  \def\hduthesis@version{0.5.0}
  \def\hduthesis@maintainerid{myhsia}
\end{minted}

调用 \pkg{etoolbox} 宏包用于给命令打补丁
\begin{minted} [ linenos, firstnumber = last, bgcolor = bg, breaklines ] {tex}
  \RequirePackage{etoolbox}
\end{minted}

提供 \cls{hduthesis} 文档类,设置文档类日期、版本号
\begin{minted} [ linenos, firstnumber = last, bgcolor = bg, breaklines ] {tex}
  \ProvidesExplClass{hduthesis} {\hduthesis@date} {\hduthesis@version}
    {LaTeX Class for Thesis at Hangzhou Dianzi University}
\end{minted}

兼容 \hologo{TeX} Live 2022 及之后的版本. 当对应命令不存在时,在已有命令基础上新增变体

\begin{minted} [ linenos, firstnumber = last, bgcolor = bg, breaklines ] {tex}
  \cs_if_exist:NF \seq_set_split:Nne
    { \cs_generate_variant:Nn \seq_set_split:Nnn { Nne } }
  \cs_if_exist:NF \seq_set_split:Nee
    { \cs_generate_variant:Nn \seq_set_split:Nnn { Nee } }
  \cs_if_exist:NF \tl_set:Ne
    { \cs_generate_variant:Nn \tl_set:Nn { Ne } }
  \cs_if_exist:NF \tl_gset:Ne
    { \cs_generate_variant:Nn \tl_gset:Nn { Ne } }
\end{minted}

定义新命令 \cs{hduthesis_msg_new:nn} 和 \cs{hduthesis_msg_error:nn} 用于新增错误消息和将错误消息广播到 Workspace
\begin{minted} [ linenos, firstnumber = last, bgcolor = bg, breaklines ] {tex}
  \cs_new_protected:Npn \hduthesis_msg_new:nn #1#2 
    { \msg_new:nnn { hduthesis } {#1} {#2} }
  \cs_new_protected:Npn \hduthesis_msg_error:nn #1#2
    { \msg_error:nnn { hduthesis } {#1} {#2} }
\end{minted}

新增错误消息:用户协议
\begin{minted} [ linenos, firstnumber = last, bgcolor = bg, breaklines ] {tex}
  \hduthesis_msg_new:nn { 用户协议 }
    { \exp_not:n
      {
        hduThesiS~ 编译受阻.~
        使用模板前请认真阅读模板说明文档封面上的「用户协议」~
        模板作者不对使用本模板产生的格式审查问题负责~
        添加选项 `agreed'[\documentclass[agreed]{hduthesis}]~
        即可顺利编译并默认代表您已同意本协议.~ 祝君科研顺利!~
        如遇问题,可邮件反馈至 xiamyphys@gmail.com.
      }
    }
\end{minted}

对命令 \cs{hduthesis_msg_error:nn} 新增 \cmd{nx} 变体
\begin{minted} [ linenos, firstnumber = last, bgcolor = bg, breaklines ] {tex}
  \cs_generate_variant:Nn \hduthesis_msg_error:nn { nx }
\end{minted}

新增错误消息:未找到模块
\begin{minted} [ linenos, firstnumber = last, bgcolor = bg, breaklines ] {tex}
  \hduthesis_msg_new:nn { not found module }
    { The~hduthesis~module~`#1'~not~found. }
\end{minted}

定义命令 \cs{hduthesis_load_module:n} 用于加载模块,若模块不存在则输出错误消息
\begin{minted} [ linenos, firstnumber = last, bgcolor = bg, breaklines ] {tex}
  \cs_new_protected_nopar:Npn \hduthesis_load_module:n #1 
    {
      \file_if_exist_input:nF { hduthesis-##1-module.code.tex }
        { \hduthesis_msg_error:nn { not found module } {##1} }
    }
\end{minted}

定义命令 \cs{hduthesis_provide_module:n} 用于提供模块
\begin{minted} [ linenos, firstnumber = last, bgcolor = bg, breaklines ] {tex}
  \cs_new_protected_nopar:Npn \hduthesis_provide_module:n #1
    {
      \ProvidesExplFile{hduthesis-#1-module.code.tex}
        {\hduthesis@date}{\hduthesis@version}
        {hduThesiS~ \text_titlecase:n {#1} ~Module}
    }
\end{minted}

定义新数组 \cs{g__hdu_base_class_options_clist},用于存储文档类选项
\begin{minted} [ linenos, firstnumber = last, bgcolor = bg, breaklines ] {tex}
  \clist_new:N \g__hdu_base_class_options_clist
\end{minted}

定义文档类选项的键:
\begin{enumerate}
  \item 布尔(Bool)值 \keys{\cmdmac~ agreed}:用户是否同意用户协议,初始值 \cmd{false}. 一旦输入 \cmd{agreed} 选项,则将 \cmd{agreed} 设置为 \cmd{true},即 \cmd{agreed} 等价于 \cmd{agreed = true}
  \item 布尔(Bool)值 \keys{\cmdmac~ l3doc}:是否启用 \cls{l3doc} 文档类,初始值 \cmd{false}. 一旦输入 \cmd{l3doc} 选项,则将 \cmd{l3doc} 设置为 \cmd{true},即 \cmd{l3doc} 等价于 \cmd{l3doc = true}
  \item 布尔(Bool)值 \keys{\cmdmac~ stationery}:是否启用 \cls{letter} 文档类,初始值 \cmd{false}. 一旦输入 \cmd{stationery} 选项,则将 \cmd{stationery} 设置为 \cmd{true},即 \cmd{stationery} 等价于 \cmd{stationery = true}
  \item 令牌(token list)\keys{\cmdmac~ math-font}:数学字体
  \item 令牌(token list) \keys{\cmdmac~ CJKmain-font}:中文主字体
  \item 令牌(token list) \keys{\cmdmac~ CJKsans-font}:中文无衬线字体
  \item 令牌(token list) \keys{\cmdmac~ CJKmono-font}:中文等宽字体
  \item 未知选项 \keys{\cmdmac~ unknown}:将未知选项交给 \cs{__hduthesis_unknown_option:n} 处理
\end{enumerate}
\begin{minted} [ linenos, firstnumber = last, bgcolor = bg, breaklines ] {tex}
  \keys_define:nn { hduthesis / classoption }
    {
      agreed .bool_set:N      = \g__hduthesis_agreement_bool,
        agreed .initial:n     = false,
        agreed .default:n     = true,
      l3doc .bool_set:N       = \g__hduthesis_doc_bool,
        l3doc .initial:n      = false,
        l3doc .default:n      = true,
      stationery .bool_set:N  = \g__hduthesis_stationery_bool,
        stationery .initial:n = false,
        stationery .default:n = true,
      math-font .tl_set:N     = \g__hduthesis_math_font,
      CJKmain-font .tl_set:N  = \g__hduthesis_main_CJK_font,
      CJKsans-font .tl_set:N  = \g__hduthesis_sans_CJK_font,
      CJKmono-font .tl_set:N  = \g__hduthesis_mono_CJK_font,
      unknown .code:n         = \__hduthesis_unknown_option:n {#1},
    }
\end{minted}

定义命令 \cs{__hduthesis_unknown_option:n} 用于处理未知选项

若未知选项为空,则将 \cs{l_keys_key_str} 加入 \cs{g__hdu_base_class_options_clist} 列表;否则设置 \cs{l_keys_key_str} 为未知选项,并将其加入 \cs{g__hdu_base_class_options_clist} 列表
\begin{minted} [ linenos, firstnumber = last, bgcolor = bg, breaklines ] {tex}
  \cs_new_protected_nopar:Npn \__hduthesis_unknown_option:n #1
    {
      \tl_if_empty:nTF { #1 }
        {
          \clist_gput_right:NV \g__hdu_base_class_options_clist \l_keys_key_str
        }
        {
          \exp_args:NNx \clist_gput_right:Nn \g__hdu_base_class_options_clist
            { \l_keys_key_str = \exp_not:n {#1} }
        }
    }
\end{minted}

处理文档类选项
\begin{minted} [ linenos, firstnumber = last, bgcolor = bg, breaklines ] {tex}
  \ProcessKeyOptions [ hduthesis / classoption ]
\end{minted}

若启用 \cmd{l3doc} 选项,加载 \cls{l3doc} 文档类为基底,加载 \file{hdu.l3doc} 模块并结束输入

\begin{minted} [ linenos, firstnumber = last, bgcolor = bg, breaklines ] {tex}
  \bool_if:NT \g__hduthesis_doc_bool
    {
      \PassOptionsToClass { 11pt, letterpaper } { l3doc }
      \exp_args:NNV \LoadClass [ \g__hdu_base_class_options_clist ] { l3doc }
      \hduthesis_load_module:n { hdu.l3doc }
      \endinput
    }
\end{minted}

若启用 \cmd{stationery} 选项,加载 \cls{letter} 文档类为基底,加载 \file{hdu.stationery} 模块并结束输入

\begin{minted} [ linenos, firstnumber = last, bgcolor = bg, breaklines ] {tex}
  \bool_if:NT \g__hduthesis_stationery_bool
    {
      \PassOptionsToClass { 12pt } { letter }
      \exp_args:NNV \LoadClass [ \g__hdu_base_class_options_clist ] { letter }
      \hduthesis_load_module:n { hdu.stationery }
      \endinput
    }
\end{minted}

为基本文档类 \cls{ctexrep} 添加选项 \keys{a4paper, zihao = -4},为 \pkg{xeCJK} 宏包添加选项 \keys{quiet, no-math},加载 \cls{ctexrep} 文档类为基底,并用 \cs{exp_args:NNV} 将未知选项展开加载到 \cls{ctexrep} 文档类上
\begin{minted} [ linenos, firstnumber = last, bgcolor = bg, breaklines ] {tex}
  \PassOptionsToClass { a4paper, zihao = -4 } { ctexrep }
  \PassOptionsToPackage { quiet, no-math } { xeCJK }
  \exp_args:NNV \LoadClass [ \g__hdu_base_class_options_clist ] { ctexrep }
\end{minted}

如果用户未确认用户协议,输出错误消息,并不加载 \pkg{hyperref} 宏包,以免因``遇到报错而停止编译''而产生额外的 \pkg{hyperref} 警告. 否则,加载 \pkg{hyperref} 宏包,设置 \pkg{hyperref} 宏包选项
\begin{minted} [ linenos, firstnumber = last, bgcolor = bg, breaklines ] {tex}
  \bool_if:NTF \g__hduthesis_agreement_bool
    {
      \RequirePackage{hyperref}
      \pdfstringdefDisableCommands
        {
          \def \cite#1 {<#1>}
          \def \hologoRobust#1 {<#1>}
        }
      \AtBeginDocument
        { \hypersetup { hidelinks, pdfproducer = hduThesiS~by~Mingyu~Xia } }
    } { \hduthesis_msg_error:nn { 用户协议 } { 未确认 } }
\end{minted}

预加载部分常用包,设置 \pkg{pgfplots} 宏包版本,设置 \pkg{hologo} 宏包字体,设置 \cs{graphicx} 包的相对路径
\begin{minted} [ linenos, firstnumber = last, bgcolor = bg, breaklines ] {tex}
  \RequirePackage { siunitx, circuitikz, pgfplots, listings,
                    hologo, lipsum, zhlipsum, booktabs, multicol }
  \pgfplotsset { compat = newest }
  \hologoFontSetup { general = \sffamily }
  \graphicspath
    {
      {./figure/}{./figures/}{./image/}{./images/}
      {./graphics/}{./graphic/}{./pictures/}{./picture/}
    }
\end{minted}

加载 \cmd{typeset} 和 \cmd{layout} 模块.
\begin{minted} [ linenos, firstnumber = last, bgcolor = bg, breaklines ] {tex}
  \hduthesis_load_module:n { typeset }
  \hduthesis_load_module:n { layout }
\end{minted}

定义命令 \cs{__hduthesis_docinfo_degree_if_aux} 用于判断学号长度,若学号长度为 8,则加载 \cmd{bc.config} 模块;否则加载 \cmd{bc.config} 或 \cmd{pg.config} 模块. 此命令将用在 \file{hduthesis-docinfo-module.code} 模块中,因为学号长度的判断必须发生在设置文档信息后

\begin{minted} [ linenos, firstnumber = last, bgcolor = bg, breaklines ] {tex}
  \cs_new_protected:Nn \__hduthesis_docinfo_degree_if_aux:
    {
      \int_compare:nNnTF { \tl_count:N \l__docinfo_stdntid_tl } = { 8 }
        { \hduthesis_load_module:n { bc.config } }
        { \hduthesis_load_module:n { pg.config } }
    }
\end{minted}

结束 \file{hduthesis.cls} 文件
\begin{minted} [ linenos, firstnumber = last, bgcolor = bg, breaklines ] {tex}
  \endinput
\end{minted}