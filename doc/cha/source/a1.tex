\subsection{\file{hduthesis.cls} 的实现}

文档类日期/版本号/开发者id
\begin{minted} [ linenos, bgcolor = bg, breaklines ] {tex}
  \def\hduthesis@date{2024/12/23}
  \def\hduthesis@version{0.5.0}
  \def\hduthesis@maintainerid{myhsia}
\end{minted}

调用 \pkg{etoolbox} 宏包用于给命令打补丁
\begin{minted} [ linenos, firstnumber = last, bgcolor = bg, breaklines ] {tex}
  \RequirePackage{etoolbox}
\end{minted}

提供 \cls{hduthesis} 文档类,设置文档类日期、版本号
\begin{minted} [ linenos, firstnumber = last, bgcolor = bg, breaklines ] {tex}
  \ProvidesExplClass{hduthesis} {\hduthesis@date} {\hduthesis@version}
    {LaTeX Class for Thesis at Hangzhou Dianzi University}
\end{minted}

兼容 \hologo{TeX} Live 2022 及之后的版本. 当对应命令不存在时,在已有命令基础上新增变体
\begin{minted} [ linenos, firstnumber = last, bgcolor = bg, breaklines ] {tex}
  \cs_if_exist:NF \seq_set_split:Nne
    { \cs_generate_variant:Nn \seq_set_split:Nnn { Nne } }
  \cs_if_exist:NF \seq_set_split:Nee
    { \cs_generate_variant:Nn \seq_set_split:Nnn { Nee } }
  \cs_if_exist:NF \tl_set:Ne
    { \cs_generate_variant:Nn \tl_set:Nn { Ne } }
  \cs_if_exist:NF \tl_gset:Ne
    { \cs_generate_variant:Nn \tl_gset:Nn { Ne } }
\end{minted}

定义新命令 \cs{hduthesis_msg_new:nn}、\cs{hduthesis_msg_error:nn} 和 \cs{hduthesis_msg_warning:nn} 用于新增错误/警告消息,并将消息广播到 Workspace.
同时为 \cs{hduthesis_msg_error:nn} 和 \cs{hduthesis_msg_warning:nn} 定义变体 \cs{hduthesis_msg_error:nx} 和 \cs{hduthesis_msg_warning:nx},用于在消息中使用参数时展开参数
\begin{minted} [ linenos, firstnumber = last, bgcolor = bg, breaklines ] {tex}
  \cs_new_protected:Npn \hduthesis_msg_new:nn #1#2 
    { \msg_new:nnn { hduthesis } {#1} {#2} }
  \cs_new_protected:Npn \hduthesis_msg_error:nn #1#2
    { \msg_error:nnn { hduthesis } {#1} {#2} }
  \cs_generate_variant:Nn \hduthesis_msg_error:nn { nx }
  \cs_new_protected:Npn \hduthesis_msg_warning:nn #1#2
    { \msg_warning:nnn { hduthesis } {#1} {#2} }
  \cs_generate_variant:Nn \hduthesis_msg_warning:nn { nx }
\end{minted}

新增错误消息 \cs{not found module}、\cs{unknown mode} 和 \cs{Users Agreement},分别对应``模块未找到''、``未知模式''和``用户协议''
\begin{minted} [ linenos, firstnumber = last, bgcolor = bg, breaklines ] {tex}
  \hduthesis_msg_new:nn { not found module }
    { The~hduthesis~module~`#1'~not~found. }
  \hduthesis_msg_new:nn { unknown mode }
    { Unknown~hduthesis~mode~`#1',~ loading~mode~`thesis'~instead. }
  \hduthesis_msg_new:nn { Users Agreement }
    { \exp_not:n
      {
        编译受阻!~
        使用模板前请阅读用户手册封面上的「用户协议」~
        !!!模板作者(@myhsia)不对使用本模板产生的格式审查问题负责!!!~
        如果您同意用户协议,在全局选项中添加 `agreed' 即可解除本错误~
        欢迎您通过邮件(myhsia@hdu.edu.cn)或GitHub提交反馈意见
      }
    }
\end{minted}

定义新命令 \cs{hduthesis_load_module:n} 用于加载模块,若模块不存在则报错
\begin{minted} [ linenos, firstnumber = last, bgcolor = bg, breaklines ] {tex}
  \cs_new_protected_nopar:Npn \hduthesis_load_module:n #1 
    {
      \clist_map_inline:nn {#1}
      {
        \file_if_exist_input:nF { hduthesis-##1-module.code.tex }
          { \hduthesis_msg_error:nn { not found module } {##1} }
      }
    }
\end{minted}

定义新命令 \cs{hduthesis_provide_module:n} 用于提供模块
\begin{minted} [ linenos, firstnumber = last, bgcolor = bg, breaklines ] {tex}
  \cs_new_protected_nopar:Npn \hduthesis_provide_module:n #1
    {
      \ProvidesExplFile{hduthesis-#1-module.code.tex}
        {\hduthesis@date}{\hduthesis@version}
        {hduThesiS~ \text_titlecase:n {#1} ~Module}
    }
\end{minted}

定义新数组 \cs{g__hdu_base_class_options_clist},用于存储文档类选项
\begin{minted} [ linenos, firstnumber = last, bgcolor = bg, breaklines, breaklines ] {tex}
  \clist_new:N \g__hdu_base_class_options_clist
\end{minted}

定义文档类选项的键:
\begin{enumerate}
  \item 布尔(Bool)值 \keys{\cmdmac~ agreed}:用户是否同意用户协议,初始值 \cmd{false}. 一旦输入 \cmd{agreed} 选项,则将 \cmd{agreed} 设置为 \cmd{true},即 \cmd{agreed} 等价于 \cmd{agreed = true}
  \item 令牌(token list)\keys{\cmdmac~ mode}:模式,可选值为 \cmd{thesis}、\cmd{l3doc}、\cmd{stationery} 和 \cmd{beamer}
  \item 令牌(token list)\keys{\cmdmac~ math-font}:数学字体
  \item 令牌(token list)\keys{\cmdmac~ CJKmain-font}:中文主字体
  \item 令牌(token list)\keys{\cmdmac~ CJKsans-font}:中文无衬线字体
  \item 令牌(token list)\keys{\cmdmac~ CJKmono-font}:中文等宽字体
  \item 未知选项 \keys{\cmdmac~ unknown}:将未知选项交给 \cs{__hduthesis_unknown_option:n} 处理
\end{enumerate}
\begin{minted} [ linenos, firstnumber = last, bgcolor = bg, breaklines ] {tex}
  \keys_define:nn { hduthesis / classoption }
    {
      agreed .bool_set:N      = \g__hduthesis_agreement_bool,
        agreed .initial:n     = false,
        agreed .default:n     = true,
      mode .tl_set:N          = \g__hduthesis_mode_tl,
      math-font .tl_set:N     = \g__hduthesis_math_font,
      CJKmain-font .tl_set:N  = \g__hduthesis_main_CJK_font,
      CJKsans-font .tl_set:N  = \g__hduthesis_sans_CJK_font,
      CJKmono-font .tl_set:N  = \g__hduthesis_mono_CJK_font,
      unknown .code:n         = \__hduthesis_unknown_option:n {#1},
    }
\end{minted}

定义新命令 \cs{__hduthesis_unknown_option:n} 用于处理未知选项. 若未知选项为空,则将 \cs{l_keys_key_str} 加入 \cs{g__hdu_base_class_options_clist} 列表;否则设置 \cs{l_keys_key_str} 为未知选项,并将其加入 \cs{g__hdu_base_class_options_clist} 列表
\begin{minted} [ linenos, firstnumber = last, bgcolor = bg, breaklines ] {tex}
  \cs_new_protected_nopar:Npn \__hduthesis_unknown_option:n #1
    {
      \tl_if_empty:nTF {#1}
      {
        \clist_gput_right:NV \g__hdu_base_class_options_clist \l_keys_key_str
      }
      {
        \exp_args:NNx \clist_gput_right:Nn \g__hdu_base_class_options_clist
          { \l_keys_key_str = \exp_not:n {#1} }
      }
    }
\end{minted}

处理文档类选项
\begin{minted} [ linenos, firstnumber = last, bgcolor = bg, breaklines ] {tex}
  \ProcessKeyOptions [ hduthesis / classoption ]
\end{minted}

若令牌 \cs{g__hduthesis_mode_tl} 为空或者不是 \cmd{thesis}、\cmd{l3doc}、\cmd{stationery} 或 \cmd{beamer},则报错``未知模式''
\begin{minted} [ linenos, firstnumber = last, bgcolor = bg, breaklines ] {tex}
  \bool_if:nT
    {
      !\str_if_empty_p:N \g__hduthesis_mode_tl &&
      !\str_if_eq_p:ee { \g__hduthesis_mode_tl } { thesis } &&
      !\str_if_eq_p:ee { \g__hduthesis_mode_tl } { l3doc } &&
      !\str_if_eq_p:ee { \g__hduthesis_mode_tl } { stationery } &&
      % !\str_if_eq_p:ee { \g__hduthesis_mode_tl } { exam } &&
      !\str_if_eq_p:ee { \g__hduthesis_mode_tl } { beamer }
    }
    { \hduthesis_msg_warning:nx { unknown mode } { \g__hduthesis_mode_tl } }
\end{minted}

若模式为 \cmd{l3doc},则使用 \cls{l3doc} 文档类,并加载 \file{hdu.l3doc} 模块
\begin{minted} [ linenos, firstnumber = last, bgcolor = bg, breaklines ] {tex}
  \str_if_eq:eeT { \g__hduthesis_mode_tl } { l3doc }
    {
      \PassOptionsToClass { 11pt, letterpaper, kernel } { l3doc }
      \exp_args:NNV \LoadClass [ \g__hdu_base_class_options_clist ] { l3doc }
      \hduthesis_load_module:n { hdu.l3doc }
      \endinput
    }
\end{minted}

若模式为 \cmd{stationery},则使用 \cls{letter} 文档类,并加载 \file{hdu.stationery} 模块
\begin{minted} [ linenos, firstnumber = last, bgcolor = bg, breaklines ] {tex}
  \str_if_eq:eeT { \g__hduthesis_mode_tl } { stationery }
    {
      \PassOptionsToClass { 12pt } { letter }
      \exp_args:NNV \LoadClass [ \g__hdu_base_class_options_clist ] { letter }
      \hduthesis_load_module:n { hdu.stationery }
      \endinput
    }
\end{minted}

% % \str_if_eq:eeT { \g__hduthesis_mode_tl } { exam }
% %   {
% %     \exp_args:NNV \LoadClass [ \g__hdu_base_class_options_clist ] { article }
% %     \hduthesis_load_module:n { hdu.exam }
% %     \endinput
% %   }

若模式为 \cmd{beamer},则使用 \cls{beamer} 文档类,并加载 \file{hdu.beamer} 模块
\begin{minted} [ linenos, firstnumber = last, bgcolor = bg, breaklines ] {tex}
  \str_if_eq:eeT { \g__hduthesis_mode_tl } { beamer }
    {
      \PassOptionsToClass { aspectratio = 2013 } { beamer }
      \exp_args:NNV \LoadClass [ \g__hdu_base_class_options_clist ] { beamer }
      \usetheme{hdu}
      \endinput
    }
\end{minted}

若模式为 \cmd{thesis} 或者是除了 \cmd{l3doc}、\cmd{stationery} 和 \cmd{beamer} 之外的未知模式,则使用 \cls{ctexrep} 文档类. 如果用户未同意用户协议,则报错``Users Agreement'';如果用户同意用户协议,则加载 \pkg{hyperref} 宏包,并对其进行设置.\footnote{\color{black!10}此举是防止因``遇到``Users Agreement''报错而停止编译产生在只进行一次编译的情况下的\pkg{hyperref}警告} 加载 \file{typeset} 和 \file{layout} 模块,根据学号长度加载 \file{bc.config} 或 \file{pg.config} 模块
\begin{minted} [ linenos, firstnumber = last, bgcolor = bg, breaklines ] {tex}
  \bool_if:nT
    {
       \str_if_eq_p:ee { \g__hduthesis_mode_tl } { thesis } ||
      !\str_if_eq_p:ee { \g__hduthesis_mode_tl } { l3doc } ||
      !\str_if_eq_p:ee { \g__hduthesis_mode_tl } { stationery } ||
      % !\str_if_eq_p:ee { \g__hduthesis_mode_tl } { exam } ||
      !\str_if_eq_p:ee { \g__hduthesis_mode_tl } { beamer }
    }
    {
      \PassOptionsToClass { a4paper, zihao = -4 } { ctexrep }
      \PassOptionsToPackage { quiet, no-math } { xeCJK }
      \exp_args:NNV \LoadClass [ \g__hdu_base_class_options_clist ] { ctexrep }
      \bool_if:NTF \g__hduthesis_agreement_bool
        {
          \RequirePackage{hyperref}
          \pdfstringdefDisableCommands
            {
              \def \cite#1 {<#1>}
              \def \hologoRobust#1 {<#1>}
            }
          \AtBeginDocument
            { \hypersetup { hidelinks, pdfproducer = hduThesiS~by~Mingyu~Xia } }
        } { \hduthesis_msg_error:nn { Users Agreement } { Unconfirmed } }
      \hduthesis_load_module:n { typeset }
      \hduthesis_load_module:n { layout }
      \cs_new_protected:Nn \__hduthesis_docinfo_degree_if_aux:
        {
          \int_compare:nNnTF { \tl_count:N \l__docinfo_stdntid_tl } = { 8 }
            { \hduthesis_load_module:n { bc.config } }
            { \hduthesis_load_module:n { pg.config } }
        }
      \endinput
    }
\end{minted}

结束 \file{hduthesis.cls} 文件
\begin{minted} [ linenos, firstnumber = last, bgcolor = bg, breaklines ] {tex}
  \endinput
\end{minted}