\subsection{\file{hduthesis-hdu.stationery-module.code} 的实现}

提供模块 \file{hdu.stationery} 文件
\begin{minted} [ linenos, bgcolor = bg, breaklines ] {tex}
  \hduthesis_provide_module:n {hdu.stationery}
\end{minted}

定义文档信息的键
\begin{minted} [ linenos, firstnumber = last, bgcolor = bg, breaklines ] {tex}
  \keys_define:nn { hdu.stationery / docinfo }
    {
      watermark .bool_set:N  = \g__docinfo_watermark_bool,
        watermark .initial:n = false,
        watermark .default:n = true,
      title .tl_set:N        = \@title,
      author .tl_set:N       = \@author,
      mail .tl_set:N         = \l__docinfo_mail_tl,
      date .tl_set:N         = \@date
    }
\end{minted}

加载所需的宏包
\begin{minted} [ linenos, firstnumber = last, bgcolor = bg, breaklines ] {tex}
  \RequirePackage{ hyperref, geometry, tikz, twemojis, fancyhdr }
  \hypersetup{hidelinks}
  \urlstyle{same}
\end{minted}

定义用户端文档信息的输入命令:清空令牌 \cs{@author} 和 \cs{l__docinfo_mail_tl} 用于后续判断,设置键值对,通过 \pkg{hyperref} 设置 PDF 信息
\begin{minted} [ linenos, firstnumber = last, bgcolor = bg, breaklines ] {tex}
  \NewDocumentCommand \DocInfo { m }
    {
      \tl_clear:N \@author
      \tl_clear:N \l__docinfo_mail_tl
      \keys_set:nn { hdu.stationery / docinfo } {#1}
      \hypersetup
        {
          pdftitle = Hangzhou Dianzi University,
          pdfsubject = \@title, pdfauthor = \@author
        }
    }
\end{minted}

定义页面布局
\begin{minted} [ linenos, firstnumber = last, bgcolor = bg, breaklines ] {tex}
  \geometry
    { hmargin = .8in, bottom = .75in, top = 1.95in, footskip = 15.87pt,
      headheight = 1.2in, headsep = .3in, footskip = .3in }
  \linespread{1.25}
  \RequirePackage [ skip = \baselineskip ] { parskip }
  \renewcommand* \familydefault { \sfdefault }
\end{minted}

定义杭州电子科技大学主题色
\begin{minted} [ linenos, firstnumber = last, bgcolor = bg, breaklines ] {tex}
  \definecolor{hdu}{HTML}{214395}
\end{minted}

定义页眉页脚样式
\begin{minted} [ linenos, firstnumber = last, bgcolor = bg, breaklines ] {tex}
  \renewcommand* \headrulewidth {2pt}
  \renewcommand* \footrulewidth {2pt}
  \hook_gput_code:nnn { cmd/headrule/before } { . } { \color{hdu!80} }
  \hook_gput_code:nnn { cmd/footrule/before } { . } { \color{hdu!80} }
  \lhead
    { {}~
      \minipage{.6\linewidth}
        \medskip \leavevmode \lower -.111in
        \hbox { \includegraphics [ height = .75in ] {hdubadge} }
      \endminipage \medskip \hfill
      \minipage{.36\linewidth}
        \medskip \vbox
          { \linespread{1.2}
            \raggedright \small \color{hdu}
            \texttwemoji{1f4cd}~ 1158~No.2~St.,~ Hangzhou,~ 310018\\
            \texttwemoji{1f4de}~ (86)~0571-86915072\\
            \texttwemoji{1f310}~ \url{www.hdu.edu.cn}
          } \medskip
      \endminipage
    }
  \lfoot
    {
      \small{}~ \texttwemoji{1f4cd}~
      1158~No.2~Street,~ Qiantang~District,~
      Hangzhou,~ Zhejiang~Province,~ 310018,~ P.R.China
    }
  \cfoot {}
  \rfoot { \small \texttwemoji{1f310}~ \url{www.hdu.edu.cn}~ }
  \pagestyle{fancy}
\end{minted}

定义标题样式. 当令牌 \cs{@author} 和 \cs{l__docinfo_mail_tl} 都不为空时,输出 ``From'';当 \cs{@author} 为空时,输出未提供作者警告信息;当 \cs{l__docinfo_mail_tl} 不为空时,输出邮箱;当 \cs{@author} 和 \cs{l__docinfo_mail_tl} 都不为空时,输出换行;输出 ``Date'' 和日期;当 \cs{@title} 为空时,输出未提供标题错误消息;输出 ``Subject'' 和标题
\begin{minted} [ linenos, firstnumber = last, bgcolor = bg, breaklines ] {tex}
  \renewcommand* \maketitle
    {
      \group_begin: \small
      \bool_if:nT
        { !\tl_if_empty_p:N \@author || !\tl_if_empty_p:N \l__docinfo_mail_tl }
        { \makebox [ 4em ] [ l ] { \scshape From } }
      \tl_if_empty:NTF \@author
        { \@latex@warning@no@line {No \noexpand \author given} } { \@author{}~ }
      \tl_if_empty:NF \l__docinfo_mail_tl
        { \texttt { <\l__docinfo_mail_tl> } }
      \bool_if:nT
        { !\tl_if_empty_p:N \@author || !\tl_if_empty_p:N \l__docinfo_mail_tl }
        { \\ }
      \makebox [ 4em ] [ l ] { \scshape Date }
        \tl_if_empty:NTF \@date \today \@date \\
      \tl_if_empty:NTF \@title
        { \@latex@error {No \noexpand \title given}\@ehc }
        { \makebox [ 4em ] [ l ] { \scshape Subject } \@title }
      \par \vspace{.5\baselineskip}
      \group_end:
    }
\end{minted}

更改字体颜色,通过钩子设置背景,添加水印
\begin{minted} [ linenos, firstnumber = last, bgcolor = bg, breaklines ] {tex}
  \AtBeginDocument { \color_select:n {black!80} }
  \DeclareHookRule { shipout / background } { hduthesis / stationery }
    { before } { pgfrcs }
  \AddToHook { shipout / background } [ hduthesis / stationery ]
    {
      \bool_if:NT \g__docinfo_watermark_bool
        {
          \tikz [ remember~picture, overlay ]
            \node [ opacity = .2 ] at (current~page)
              { \includegraphics [ width = .4\linewidth ] {hdulogo} };
        }
    }
\end{minted}

加载宏包 \pkg{tikzpagenodes} 用于定位文本区域. 定义命令 \cs{notelines} 用于绘制线条,参数为行数,默认为 20 行. 使用 \pkg{tikz} 绘制线条,透明度为 0.6,颜色为 \cs{hdu},线宽为 2pt,从当前页文本区域的西北角开始,向东绘制线条 \cs{linewidth} 长,每行间隔为 \cs{textheight} 的 20 分之一
\begin{minted} [ linenos, firstnumber = last, bgcolor = bg, breaklines ] {tex}
  \RequirePackage{tikzpagenodes}
  \NewDocumentCommand \notelines { O{20} }
    {
      \tikz [ remember~picture, overlay ]
        {
          \int_step_inline:nn { #1 - 1 }
            {
              \draw [ hdu, very~thick, opacity = .6 ]
                ([yshift = -##1 * (\textheight + .6in - 15.87pt ) / #1 + .3in]
                  current~page~text~area.north~west) --++ (\linewidth, 0);
            }
        }
    }
\end{minted}

从 \cls{ltxdoc} 中抄取命令 \cs{meta}
\begin{minted} [ linenos, firstnumber = last, bgcolor = bg, breaklines ] {tex}
  \newcommand \meta[1]
    {
      \ensuremath \langle
        \ifmmode \expandafter \nfss@text \fi
        {
          \itshape\ttfamily \edef \meta@hyphen@restore
            { \hyphenchar \the \font \the \hyphenchar \font }
          \hyphenchar \font \m@ne \language
          \l@nohyphenation #1\/\meta@hyphen@restore
        }
      \ensuremath \rangle 
    }
\end{minted}

结束模块 \file{hduthesis-stationery-module.code} 文件
\begin{minted} [ linenos, firstnumber = last, bgcolor = bg, breaklines ] {tex}
  \endinput
\end{minted}
