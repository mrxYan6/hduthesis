\subsection{\file{hduthesis-hdu.stationery-module.code} 的实现}

模块 \file{stationery} 的代码实现与之前类似. 不同的部分有使用钩子 \cmd{shipout} 来实现水印的添加

\begin{minted} [ linenos, firstnumber = last, bgcolor = bg, breaklines ] {tex}
  \DeclareHookRule { shipout / background } { hduthesis / stationery }
    { before } { pgfrcs }
  \AddToHook { shipout / background } [ hduthesis / stationery ]
    {
      \bool_if:NT \g__docinfo_watermark_bool
        {
          \tikz [ remember~picture, overlay ]
            \node [ opacity = .2 ] at (current~page)
              { \includegraphics [ width = .4\linewidth ] {hdulogo} };
        }
    }
\end{minted}

调用 \pkg{tikzpagenodes} 宏包用于定位版心的边缘,用循环句柄 \cs{int_step_inline:nn} 与 \cs{draw} 命令来实现横线的绘制.

\begin{minted} [ linenos, firstnumber = last, bgcolor = bg, breaklines ] {tex}
  \RequirePackage{tikzpagenodes}
  \NewDocumentCommand \notelines {O{20}}
    {
      \tikz [ remember~picture, overlay ]
        {
          \int_step_inline:nn { #1 - 1 }
            {
              \draw [ hdu, very~thick, opacity = .6 ]
                ([yshift = -##1 * (\textheight + .6in - 15.87pt ) / #1 + .3in]
                  current~page~text~area.north~west) --++ (\linewidth, 0);
            }
        }
    }
\end{minted}
