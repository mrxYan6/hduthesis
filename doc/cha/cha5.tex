\chapter{软件设计}

\section{总体方案}

该系统的信号检测与数据传送部分,涉及的软件部分较少。主要是 2051 单片机数据串行通
信及通信协议的程序设计。

对于 2051 的程序设计\cite{cn12},由于所需实现的功能较简单,采用汇编的形式。编
译器采用 Keil 7.02b。该编译器是 51 系列单片机程序设计的常用工具,既可用汇编,
也支持 C 语言编译。同时具有完善的调试功能。

\section{程序流图}

\begin{figure}[htbp]
  \centering
  \begin{tikzpicture}
  [
    every node/.style = { font = {\small}, minimum height = 2.5em}
  ]
    \node [draw, rectangle, minimum width = 12em, minimum height = 2em] (a) at (0,0) {初始参数设置};
    \node [draw, rectangle, minimum width = 10em, rounded corners = 1.2em, below = of a] (b) {等待中断};
    \node [draw, rectangle, minimum width = 11em, diamond, aspect=3, below = of b] (c) {中断服务程};
    \node [draw, rectangle, minimum width = 11em, diamond, aspect=3, below = of c, yshift = 2ex] (d) {读取 P1 口值};
    \node [draw, rectangle, minimum width = 8em, rectangle, below = of d, minimum height = 2em] (e) {发送数据帧};
    \node [draw, rectangle, minimum width = 11em, diamond, aspect=3, below = of e] (f) {延时$\qty{200}\ms$};
    \node [draw, rectangle, minimum width = 8em, rectangle, below = of f, minimum height = 2em, yshift = 2ex] (g) {清中断标志};
    \node [draw, rectangle, minimum width = 10em, rounded corners = 1.2em, below = of g] (h) {中断返回};
    \draw [->] (a.south) -- (b.north);
    \draw [->] (b.south) -- (c.north);
    \draw [->] (c.south) -- (d.north);
    \draw [->] (d.south) -- (e.north);
    \draw [->] (e.south) -- (f.north);
    \draw [->] (f.south) -- (g.north);
    \draw [->] (g.south) -- (h.north);
    \draw [->] (h.east) --++ (1,0) |- (b.east);
  \end{tikzpicture}
  \caption{串行发送流程图}
  \label{5-1}
\end{figure}

\section{模块说明}
\newpage

\begin{enumerate}
  \item 主程序:
  \begin{lstlisting}[basicstyle=\linespread{1.3}\selectfont, breaklines=true]
      MAIN:
      MOV SP,#0X60  ;堆栈初始化
      CALL INIT     ;各寄存器参数设置
      MOV 40H,#0x01 ;打靶次数置 1
      AJMP $        ;等待中断
  \end{lstlisting}
  \item 初始化程序:
  \begin{lstlisting}[basicstyle=\linespread{1.3}\selectfont, breaklines=true]
      INIT:
      MOV TMOD,#0X21;波特率发生器
      MOV TL1,#0XFD ;波特率 9600bps
      MOV TH1,#0XFD
      CLR ET1       ;禁止 timer1
      SETB PT1      ;时钟 1 优先级:高
      MOV SCON,#0x40;串口工作模式 1,SM2=0,REN=0
      MOV PCON,#0   ;波特率 9600bps
      SETB EA       ;中断允许
      CLR PS        ;关闭串口中断
      CLR ES        ;串口优先级:低
      SETB EX0      ;开外部中断 INT0 SETB IT0 ;下降沿有效
      CLR PX0       ;INT0 优先级:低
      SETB TR1      ;时钟 1 开始计数
      RET
  \end{lstlisting}
  \item 中断服务程序:
  \begin{lstlisting}[basicstyle=\linespread{1.3}\selectfont, breaklines=true]
      _INT0:        ;ISR 中断服务程序
      NOP
      CALL DELAY_2MS;同步延时
      MOV P1,#0xff  ;读 P1 口前先置 1
      MOV A,P1      ;读 P1 口
      CALL INT0_SEND
      RET
  \end{lstlisting}
  \item 数据帧传送子程序:
  \begin{figure}[htbp]
    \centering\small
    \renewcommand{\arraystretch}{.6}
    \caption{数据帧格式}
    \begin{tabular}{|*{4}{>{\centering\arraybackslash}p{.22\linewidth}|}}
      \hline
      标志位SYNC & 打靶次数 & 打靶成绩 & 校验位CHECKSUM\\
      \hline
      \#0x30 & TIMES & RESULT & TIMES\ensuremath+RES\\
      \hline
    \end{tabular}
  \end{figure}

  例:30 02 15 17(十六进制)

  表示第二次打靶,击中第 21 号(对应环数:7 环 偏移方向:右上)。
  \begin{lstlisting}[basicstyle=\linespread{1.3}\selectfont, breaklines=true]
      INT0_SEND:      ;数据帧传送子程序
      PUSH ACC        ;保护 ACC
      CLR A
      ADD A,#0X30
      CALL UART_SEND  ;发送标志位
      MOV A,40H
      CALL UART_SEND  ;发送打靶次数
      POP ACC
      CALL UART_SEND  ;发送打靶成绩
      ADD A,#0X30
      ADD A,0040H
      CALL UART_SEND  ;发送校验位
      INC 0040H       ;打靶次数累加 1
      CALL DELAY_200MS;延时 200ms
      CLR EX0         ;关外部中断
      CLR IE0         ;清 INT0 外部中断请求标志位—防止外部中断寄存而引起多次中断。
      SETB EX0        ;开中断
      RETI
  \end{lstlisting}
  \item 串行发送字节
  \begin{lstlisting}[basicstyle=\linespread{1.3}\selectfont, breaklines=true]
      UART_SEND:     ;串行发送一个字节
      MOV SBUF,A
      JNB TI,$       ;等待发送完毕
      CLR TI         ;
      RET
  \end{lstlisting}
  \item 定时程序:
  \begin{lstlisting}[basicstyle=\linespread{1.3}\selectfont, breaklines=true]
      DELAY_2MS:     ;用定时器延时 2ms
      MOV R7,#21;21
      DLY1:MOV R6,#42
      DLY2:DJNZ R6,DLY2
      DJNZ R7,DLY1
      RET
      DELAY_10MS:   ;调用 DELAY_2MS,实现延时 10ms
      MOV R5,#5
      DLY: CALL DELAY_2MS
      DJNZ R5,DLY
      RET
\end{lstlisting}
\end{enumerate}