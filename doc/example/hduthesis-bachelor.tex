\documentclass
  [
    math-font    = STIX Two Math, agreed,
    CJKmain-font = { {Songti SC}[AutoFakeBold = 2.5, AutoFakeSlant] },
    CJKsans-font = { {STHeiti}[AutoFakeBold = 2] }
  ] {hduthesis}
\tikzset{ > = stealth }
\usetikzlibrary{positioning, shapes.geometric}

\hduset
  {
    title      = HDU 学士学位论文 \hologo{LaTeX} 模板示例文档/
                 本科毕业设计,
    department = 理学院,
    major      = 物理学,
    class      = 英才班,
    stdntid    = C668668E,
    author     = 申智能,
    supervisor = 教授:葉芷晴,
    % bibsource  = xampl
  }

\begin{document}

\maketitle
\commitment [ example-image-a/2024-05-31 ]

\begin{abstract}[cn]
  \keywords{杭州电子科技大学, 毕业论文, \hologo{LaTeX3}, }
\end{abstract}

\begin{abstract}[en]
  \keywords{HDU, thesis, \hologo{LaTeX3}, }
\end{abstract}

\tableofcontents

\chapter{引言}
1

\cite{whole-journal}

\zhlipsum[2]
\cite{inbook-crossref,whole-set,booklet-full,incollection-crossref,whole-collection,manual-full,mastersthesis-full}

\chapter{公式与插图测试}

\begin{equation}
  (i\gamma^\mu\partial_\mu - m)\psi = 0
  \label{2-1}
\end{equation}
其中 \eqref{2-1} 为Dirac 方程.
\[
  S_x = \frac12\sigma_1 = \frac12
  \begin{pmatrix}
    0 & 1\\
    1 & 0
  \end{pmatrix}, \quad
  S_y = \frac12\sigma_1 = \frac12
  \begin{pmatrix}
    0 & -i\\
    i & 0
  \end{pmatrix}, \quad
  S_z = \frac12\sigma_1 = \frac12
  \begin{pmatrix}
    1 & 0\\
    0 & -1
  \end{pmatrix}
\]
上式为 Pauli 矩阵.

\printbibliography

\appendix
\chapter*{附录}
\addcontentsline{toc}{chapter}{附录}

\end{document}
